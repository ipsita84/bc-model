\documentclass[aps,prl,superscriptaddress,twocolumn]{revtex4}

\pdfoutput=1


\usepackage{amsmath}
\usepackage{graphicx}
\usepackage{color}




\newcommand{\note}[1]{%
  \marginpar{\vskip-\baselineskip\raggedright\tiny\sffamily\hrule\smallskip%
      {\color{red}#1}\par\smallskip\hrule}}

\bibliographystyle{apsrev}

\begin{document}

\newcommand{\beq}{\begin{equation}}
\newcommand{\eeq}{\end{equation}}
\newcommand{\beqa}{\begin{eqnarray}}
\newcommand{\eeqa}{\end{eqnarray}}
\newcommand{\ben}{\begin{enumerate}}
\newcommand{\een}{\end{enumerate}}
\newcommand{\hs}{\hspace{0.5cm}}
\newcommand{\vs}{\vspace{0.5cm}}

\title{Detecting the Tricritical Point of the 2$d$ Blume--Capel Model}

\author{Ipsita Mandal}
\affiliation{Perimeter Institute for Theoretical Physics, Waterloo, Ontario N2L 2Y5, Canada}

\author{Stephen Inglis}
\affiliation{Department of Physics and Arnold Sommerfeld
Center for Theoretical Physics, Ludwig-Maximilians-Universit\"at
M\"unchen, D-80333 M\"unchen, Germany}


\author{Roger G. Melko}
\affiliation{Department of Physics and Astronomy, University of Waterloo, Ontario, N2L 3G1, Canada}
\affiliation{Perimeter Institute for Theoretical Physics, Waterloo, Ontario N2L 2Y5, Canada}

\date{\today}

\begin{abstract}
The spin-1 two-dimensional classical Blume-Capel model on a square lattice 
is known to exhibit a tricritical point
described by the tricritical Ising CFT with central charge $c=7/10$.
By using the Renyi entropies via a replica-trick 
on classical statistical mechanical systems,
and calculating the Renyi Mutual Information (RMI) with Monte Carlo simulations,
we can extract the value of the central charge at the tricritical point. We vary the parameters of
Hamiltonian such that we obtain the tricritical point reproducing the
correct central charge value predicted by CFT.
\end{abstract}

\maketitle

{\em Introduction --}
The Hamiltonian of the
spin-1 Blume-Capel model on a two-dimensional square lattice \cite{blume,capel} is given by
%%%%%%%%%%%%%
\beq 
H = -J \sum_{\langle i j \rangle} S^z_i S^z_j
+ D \sum_{\langle i  \rangle} ( S ^z_i) ^2   ,
\label{bc-model}
\eeq
%%%%%%%%%%%%%%%
where $S^z_i = \pm 1, 0 $.
Though the model cannot be solved exactly, it has been shown to exhibit a tricritical point
described by the tricritical Ising CFT with central charge $c=7/10$ \cite{balbao}. However, the position of the
tricritical point can be found only numerically as this is not an exactly solvable model. There has been
an extensive study in the literature using various sophisticated numerical techniques \cite{burk,berker,burk2,ng,landau,selke,chak,beale,landau2,tucker,du,du2,silva,yusuf} to pin down the values of the parameters $D , \, J$ and temperature $ T $ of the tricritical point. Here we endeavor to detect this
phase transition point by using the quantity called Renyi Mutual Information (RMI), which is able to detect all correlations in a physical system, even those
missed by traditional connected correlation functions. The method of detection of detection of classical phase transitions using RMI was developed by two of the authors \citep{stephen2013,stephen2014}.
Since the RMI can detect finite-temperature critical points, and even identify their universality class,
without the knowledge of an order parameter or other thermodynamic estimators, if we are sitting exactly at
the tricirtical point, we expect to extract a value of $c$ from our numerical simulations, matching with the CFT value of $0.7$.


{\em Results --}
Varying $\left ( D_c /J , \, k T_c / J \right ) $ values, we find that the closest match to the actual $c=0.7$ is obtained for $\left ( D_c /J =1.9 , \, k T_c / J =0.79 \right ) $. We also provide a $\chi^2$ estimate, which clearly
indicates this as the best fit.







{\em Discussion --}




{\em Acknowledgements --} We thank ... for enlightening discussions.  
 This work was made possible by the computing facilities of SHARCNET. Support was provided 
by NSERC of Canada (I.M. and R.G.M.), the Templeton
Foundation (I.M.) and the National Science Foundation under Grant No. NSF PHY11-25915 (R.G.M). Research at the Perimeter Institute is supported, in
part, by the Government of Canada through Industry Canada
and by the Province of Ontario through the Ministry of
Research and Information.


\bibliography{bc-draft}

\end{document}
